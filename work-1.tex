\documentclass[14pt]{extarticle}
\usepackage[a4paper,
            left=3cm,
            right=1.5cm,
            top=2cm,
            bottom=2cm,
            includehead,includefoot,headheight=0pt,%showframe,
            nohead, nofoot, nomarginpar, 
            ]{geometry}
\linespread{1.5}
\usepackage{silence}
\WarningsOff[fancyhdr]
\usepackage{fancyhdr}
\usepackage{setspace}
\usepackage{listings}
\usepackage{booktabs}
\usepackage{siunitx}
\usepackage[utf8]{inputenc}
\usepackage[T1,T2A]{fontenc}
\usepackage[english, russian]{babel}
\usepackage{amsmath,mathtools,amssymb}
\fontsize{14}{17.3}\selectfont
\usepackage[table,dvipsnames]{xcolor}
\usepackage{soulutf8}
\usepackage{multirow}
\definecolor{LightGray}{rgb}{0.95,0.95,0.95}
\newcommand\college{Томский техникум информационных технологий}
%%%%%%%%%%%%%%%%%%%%%%%%%%%%%%%%%%%%%%%%%%%%%%%%%%%%%%%%%%%%%%%%%%%%%%%%%%%%%%%%
\definecolor{codegreen}{rgb}{0,0.6,0}
\definecolor{codegray}{rgb}{0.5,0.5,0.5}
\definecolor{codepurple}{rgb}{0.18,0,0.82}
\definecolor{backcolour}{rgb}{0.95,0.95,0.92}
\lstdefinestyle{mystyle}{
	backgroundcolor=\color{backcolour},
	commentstyle=\color{codegreen},
	keywordstyle=\color{codepurple},
	numberstyle=\tiny\color{codegray},
	stringstyle=\color{codepurple},
	basicstyle=\ttfamily\footnotesize,
	breakatwhitespace=false,
	breaklines=true,
	captionpos=b,
	keepspaces=true,
	numbers=left,
	numbersep=5pt,
	showspaces=false,
	showstringspaces=false,
	showtabs=false,
	tabsize=2,
	numbers=none,
	framesep=10pt,
	xleftmargin=7pt,
	xrightmargin=7pt,
	framexleftmargin=7pt,
	framexrightmargin=7pt
}
\lstset{style=mystyle}

\usepackage{enumitem}

\usepackage{listings}

\setlength\parindent{1.25cm}
\setlength{\footskip}{30pt}
\usepackage{indentfirst}

%\counterwithin{figure}{chapter}
%\counterwithin{table}{chapter}

\usepackage{titlesec}
\titleformat{\section}{\normalfont\normalsize\filcenter}{\thesection}{0.3em}{}
\titleformat{\subsection}{\normalfont\normalsize}{\thesubsection}{0.3em}{}
\titleformat{\subsubsection}{\normalfont\normalsize}{\thesubsubsection}{0.3em}{}

\titlespacing*{\section}{0pt}{0pt}{20pt}
\titlespacing*{\subsection}{1.25cm}{0pt}{20pt}
\titlespacing*{\subsubsection}{1.25cm}{0pt}{20pt}

\newlist{mylist}{itemize}{1}
\setlist[mylist]{label=—, leftmargin=1.9, rightmargin=0pt}

\newlist{listNoSpaceAfter}{enumerate}{1}
\setlist[listNoSpaceAfter]{label=\arabic*), leftmargin=0cm, itemindent=1.9cm, noitemsep, before=\vspace{-16px}, after=\vspace{-20px}}

\setlist[enumerate]{label=\arabic*), leftmargin=0cm, itemindent=1.9cm, noitemsep, before=\vspace{-16px}, after=\vspace{-12px}}

\usepackage{fontspec}
\setmainfont{pt-astra-serif-regular.ttf}

\usepackage{unicode-math}
\defaultfontfeatures[pt-astra-serif-regular]
    { %Path = {/},
      Extension = .ttf }

\babelfont{tt}{Cousine}

\makeatletter
\renewcommand*\normalsize{%
  \@setfontsize\normalsize{14}{21}
}
\makeatother

\usepackage{caption}
\DeclareCaptionLabelSeparator{emdash}{\space\textemdash\space}
\captionsetup[figure]{name={Рисунок}, labelsep=endash}
\captionsetup[table]{name={Таблица}, labelsep=endash}
\captionsetup[listing]{name={Листинг}, labelsep=endash}
\captionsetup[table]{skip=0pt, singlelinecheck=off}

\usepackage{caption}
\usepackage[newfloat]{minted}
\usepackage{ragged2e}
\captionsetup[lstlisting]{position=top, labelsep=endash, justification=justified, singlelinecheck=off}

\fancyhfoffset[E,O]{0pt}

\setlength{\belowdisplayskip}{0pt} \setlength{\belowdisplayshortskip}{0pt}
\setlength{\abovedisplayskip}{0pt} \setlength{\abovedisplayshortskip}{0pt}

\newcommand\variant[1]{#1}
\newcommand\answer[2]{#2}
\newcommand\Group{600}
\newcommand\FirstStudent{И.\,И. Иванов}
%\newcommand\SecondStudent{П.\,П. Петров}
\newcommand\Type{Проверочная работа}
\newcommand\Topic{Представление целых чисел}
\newcommand\integer[4]{\begin{center}\vspace{-1.0em}\texttt{#1} \texttt{#2} \texttt{#3} \texttt{#4}\end{center}\vspace{-0.5em}}

\begin{document}
\pagestyle{fancy}
\chead{}
\cfoot{}
\lhead{}
\rhead{}
\renewcommand{\headrulewidth}{0pt}
\renewcommand{\footrulewidth}{0pt}
\cfoot{\thepage}
\thispagestyle{empty}

\setmathfont[range=up/{num}]{pt-astra-serif-regular.ttf}
\setmathfont[range=it/{latin,Latin}]{pt-astra-serif-italic.ttf}


\begin{spacing}{1.5}
\begin{center}
ДЕПАРТАМЕНТ ОБРАЗОВАНИЯ ТОМСКОЙ ОБЛАСТИ\vspace{0px}

ОБЛАСТНОЕ ГОСУДАРСТВЕННОЕ БЮДЖЕТНОЕ\vspace{-10px}\\ 
ПРОФЕССИОНАЛЬНОЕ ОБРАЗОВАТЕЛЬНОЕ УЧРЕЖДЕНИЕ\vspace{-10px}\\
\raisebox{.14ex}{«}ТОМСКИЙ ТЕХНИКУМ ИНФОРМАЦИОННЫХ ТЕХНОЛОГИЙ\raisebox{.14ex}{»}

\vspace{0.5cm}

Специальность 09.02.07 Информационные системы и программирование

\vspace{4cm}

\Type\vspace{-10px}\\по дисциплине «Численные методы»

\MakeUppercase{\Topic}

\end{center}


\vspace{3cm}

Студент\ifdefined\SecondStudent ы \fi~гр. \Group\vspace{-5px}

«\rule{5mm}{0.15mm}» \rule{20mm}{0.15mm} \the\year~г.
\hspace{16mm} \rule{35mm}{0.15mm}
\hspace{2mm} \FirstStudent

\ifdefined\SecondStudent

«\rule{5mm}{0.15mm}» \rule{20mm}{0.15mm} \the\year~г.
\hspace{16mm} \rule{35mm}{0.15mm}
\hspace{2mm} \SecondStudent

\fi

\vspace{-10px}

Преподаватель\vspace{-5px}

«\rule{0.5cm}{0.15mm}» \rule{2cm}{0.15mm} \the\year~г.
\hspace{16mm} \rule{35mm}{0.15mm}
\hspace{2mm} Д.\,И. Жабин


\vspace{2cm}

\begin{center}
Томск \the\year{}
\end{center}
\newpage
\end{spacing}

\cfoot{\thepage}

%%%%%%%%%%%%%%%%%%%%%%%%%%%%%%%%%%%%%%%%%%%%%%%%%%%%%%%%%%%%%%%%%%%%%%%%%%%%%%%%

\section{\MakeUppercase{Постановка задачи}}

Цель работы: ознакомиться с представлением в памяти целых чисел.

Задачи:

\begin{enumerate}[]

    \item записать представление двух чисел: положительного --- в виде прямого кода, отрицательного --- в виде дополнения до двух;
    
    \item выполнить сложение двух чисел;
    
    \item проверить результат вычислений.

\end{enumerate}

Вариант №\,30. Сложить числа $7$ и $-5$, записанные в памяти компьютера как 32-битные целые.

\newpage

\section{\MakeUppercase{Ход работы}}

Записываем число $7$ в~виде 32-разрядного прямого кода.

\answer{1}{\integer{00000000}{00000000}{00000000}{00000111}}

Записываем число $-5$ в~виде 32-разрядного дополнения до~двух. Прямой код числа $-5$ --- это число $5$ в~двоичной системе:

\answer{2a}{\integer{00000000}{00000000}{00000000}{00000101}}

Обратный код числа --- это инвертированный прямой код:
\answer{2b}{\integer{11111111}{11111111}{11111111}{11111010}}

Дополнительный код --- это результат арифметического сложения обратного кода
с~1:

\answer{2c}{\integer{11111111}{11111111}{11111111}{11111011}}

Выполняем вычисления: $7 + (-5).$

Нужно сложить прямой код числа 7 и дополнительный код числа $-5.$

\begin{center}
	\begin{tabular}{c@{\,}r}
		  &  \texttt{00000000000000000000000000000111} \\
		+ &  \texttt{11111111111111111111111111111011} \\
		  \hline
		  & \texttt{{\color{lightgray}1}00000000000000000000000000000010} \\
	\end{tabular}
\end{center}

Полученный результат (левая единица отбрасывается, так как не помещается
в~отведённые 32 бита):

\answer{3a}{\integer{00000000}{00000000}{00000000}{00000010}}

Первый бит равен 0, это значит, что число положительное. Таким образом,
результат сложения — это число $10_2 = \answer{3b}{2}_{10}.$

\newpage

\section{\MakeUppercase{Заключение}}

В ходе работы мы ознакомились с представлением в памяти компьютера целых чисел. Мы выполнили поставленные задачи: записали представление двух чисел, сложили их и проверили результат.

\end{document}
