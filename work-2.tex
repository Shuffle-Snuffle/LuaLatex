\documentclass[14pt]{extarticle}
\usepackage[a4paper,
            left=3cm,
            right=1.5cm,
            top=2cm,
            bottom=2cm,
            includehead,includefoot,headheight=0pt,%showframe,
            nohead, nofoot, nomarginpar, 
            ]{geometry}
\linespread{1.5}
\usepackage{silence}
\WarningsOff[fancyhdr]
\usepackage{fancyhdr}
\usepackage{setspace}
\usepackage{listings}
\usepackage{booktabs}
\usepackage{siunitx}
\usepackage[utf8]{inputenc}
\usepackage[T1,T2A]{fontenc}
\usepackage[english, russian]{babel}
\usepackage{amsmath,mathtools,amssymb}
\fontsize{14}{17.3}\selectfont
\usepackage[table,dvipsnames]{xcolor}
\usepackage{soulutf8}
\usepackage{multirow}
\definecolor{LightGray}{rgb}{0.95,0.95,0.95}
\newcommand\college{Томский техникум информационных технологий}
%%%%%%%%%%%%%%%%%%%%%%%%%%%%%%%%%%%%%%%%%%%%%%%%%%%%%%%%%%%%%%%%%%%%%%%%%%%%%%%%
\definecolor{codegreen}{rgb}{0,0.6,0}
\definecolor{codegray}{rgb}{0.5,0.5,0.5}
\definecolor{codepurple}{rgb}{0.18,0,0.82}
\definecolor{backcolour}{rgb}{0.95,0.95,0.92}
\lstdefinestyle{mystyle}{
	backgroundcolor=\color{backcolour},
	commentstyle=\color{codegreen},
	keywordstyle=\color{codepurple},
	numberstyle=\tiny\color{codegray},
	stringstyle=\color{codepurple},
	basicstyle=\ttfamily\footnotesize,
	breakatwhitespace=false,
	breaklines=true,
	captionpos=b,
	keepspaces=true,
	numbers=left,
	numbersep=5pt,
	showspaces=false,
	showstringspaces=false,
	showtabs=false,
	tabsize=2,
	numbers=none,
	framesep=10pt,
	xleftmargin=7pt,
	xrightmargin=7pt,
	framexleftmargin=7pt,
	framexrightmargin=7pt
}
\lstset{style=mystyle}

\usepackage{enumitem}

\usepackage{listings}

\setlength\parindent{1.25cm}
\setlength{\footskip}{30pt}
\usepackage{indentfirst}

%\counterwithin{figure}{chapter}
%\counterwithin{table}{chapter}

\usepackage{titlesec}
\titleformat{\section}{\normalfont\normalsize\filcenter}{\thesection}{0.3em}{}
\titleformat{\subsection}{\normalfont\normalsize}{\thesubsection}{0.3em}{}
\titleformat{\subsubsection}{\normalfont\normalsize}{\thesubsubsection}{0.3em}{}

\titlespacing*{\section}{0pt}{0pt}{20pt}
\titlespacing*{\subsection}{1.25cm}{0pt}{20pt}
\titlespacing*{\subsubsection}{1.25cm}{0pt}{20pt}

\newlist{mylist}{itemize}{1}
\setlist[mylist]{label=—, leftmargin=1.9, rightmargin=0pt}

\newlist{listNoSpaceAfter}{enumerate}{1}
\setlist[listNoSpaceAfter]{label=\arabic*), leftmargin=0cm, itemindent=1.9cm, noitemsep, before=\vspace{-16px}, after=\vspace{-20px}}

\setlist[enumerate]{label=\arabic*), leftmargin=0cm, itemindent=1.9cm, noitemsep, before=\vspace{-16px}, after=\vspace{-12px}}

\usepackage{fontspec}
\setmainfont{pt-astra-serif-regular.ttf}

\usepackage{unicode-math}
\defaultfontfeatures[pt-astra-serif-regular]
    { %Path = {/},
      Extension = .ttf }

\babelfont{tt}{Cousine}

\makeatletter
\renewcommand*\normalsize{%
  \@setfontsize\normalsize{14}{21}
}
\makeatother

\usepackage{caption}
\DeclareCaptionLabelSeparator{emdash}{\space\textemdash\space}
\captionsetup[figure]{name={Рисунок}, labelsep=endash}
\captionsetup[table]{name={Таблица}, labelsep=endash}
\captionsetup[listing]{name={Листинг}, labelsep=endash}
\captionsetup[table]{skip=0pt, singlelinecheck=off}

\usepackage{caption}
\usepackage[newfloat]{minted}
\usepackage{ragged2e}
\captionsetup[lstlisting]{position=top, labelsep=endash, justification=justified, singlelinecheck=off}

\fancyhfoffset[E,O]{0pt}

\setlength{\belowdisplayskip}{0pt} \setlength{\belowdisplayshortskip}{0pt}
\setlength{\abovedisplayskip}{0pt} \setlength{\abovedisplayshortskip}{0pt}

\newcommand\variant[1]{#1}
\newcommand\answer[2]{#2}
\newcommand\Group{600}
\newcommand\FirstStudent{И.\,И. Иванов}
%\newcommand\SecondStudent{П.\,П. Петров}
\newcommand\Type{Проверочная работа}
\newcommand\Topic{Представление чисел с плавающей запятой}
\newcommand\fp[3]{\begin{center}\vspace{-1.0em}\texttt{
        {\color{Maroon}{#1}}
        {\color{ForestGreen}{#2}}
        {\color{Cerulean}{#3}}
    }\end{center}\vspace{-0.5em}}

\begin{document}
\pagestyle{fancy}
\chead{}
\cfoot{}
\lhead{}
\rhead{}
\renewcommand{\headrulewidth}{0pt}
\renewcommand{\footrulewidth}{0pt}
\cfoot{\thepage}
\thispagestyle{empty}

\setmathfont[range=up/{num}]{pt-astra-serif-regular.ttf}
\setmathfont[range=it/{latin,Latin}]{pt-astra-serif-italic.ttf}


\begin{spacing}{1.5}
\begin{center}
ДЕПАРТАМЕНТ ОБРАЗОВАНИЯ ТОМСКОЙ ОБЛАСТИ\vspace{0px}

ОБЛАСТНОЕ ГОСУДАРСТВЕННОЕ БЮДЖЕТНОЕ\vspace{-10px}\\ 
ПРОФЕССИОНАЛЬНОЕ ОБРАЗОВАТЕЛЬНОЕ УЧРЕЖДЕНИЕ\vspace{-10px}\\
\raisebox{.14ex}{«}ТОМСКИЙ ТЕХНИКУМ ИНФОРМАЦИОННЫХ ТЕХНОЛОГИЙ\raisebox{.14ex}{»}

\vspace{0.5cm}

Специальность 09.02.07 Информационные системы и программирование

\vspace{4cm}

\Type\vspace{-10px}\\по дисциплине «Численные методы»

\MakeUppercase{\Topic}

\end{center}


\vspace{3cm}

Студент\ifdefined\SecondStudent ы \fi~гр. \Group\vspace{-5px}

«\rule{5mm}{0.15mm}» \rule{20mm}{0.15mm} \the\year~г.
\hspace{16mm} \rule{35mm}{0.15mm}
\hspace{2mm} \FirstStudent

\ifdefined\SecondStudent

«\rule{5mm}{0.15mm}» \rule{20mm}{0.15mm} \the\year~г.
\hspace{16mm} \rule{35mm}{0.15mm}
\hspace{2mm} \SecondStudent

\fi

\vspace{-10px}

Преподаватель\vspace{-5px}

«\rule{0.5cm}{0.15mm}» \rule{2cm}{0.15mm} \the\year~г.
\hspace{16mm} \rule{35mm}{0.15mm}
\hspace{2mm} Д.\,И. Жабин


\vspace{2cm}

\begin{center}
Томск \the\year{}
\end{center}
\newpage
\end{spacing}

\cfoot{\thepage}

%%%%%%%%%%%%%%%%%%%%%%%%%%%%%%%%%%%%%%%%%%%%%%%%%%%%%%%%%%%%%%%%%%%%%%%%%%%%%%%%

\section{\MakeUppercase{Постановка задачи}}

Цель работы: ознакомиться с представлением в памяти чисел с плавающей запятой.

Задачи:

\begin{enumerate}[]

    \item записать число $N_1$ в формате с плавающей запятой;
    
    \item по побитовому представлению числа с плавающей запятой $N_2$ определить, что это было за число;
    
    \item вычислить $N_1 \cdot N_2$.

\end{enumerate}

Вариант №\,30. $N_1 = 3{,}5$, представление числа $N_2$:

\fp{0}{10000011}{11010010000000000000000}

\newpage

\section{\MakeUppercase{Ход работы}}

%%%%%%%%%%%%%%%%%%%%%%%%%%%%%%%%%%%%%%%%%%%%%%%%%%%%%%%%%%%%%%%%%%%%%%%%%%%%%%%%

Общая форма числа с плавающей запятой:
$N = (-1)^{\color{Maroon} s} \cdot {\color{Cerulean} m}
\cdot 2^{\color{YellowGreen} e}$.
Первый бит~--- бит знака. Поскольку число положительное,
$s = {\color{Maroon} 0}$. Чтобы определить мантиссу и порядок, переведём число в двоичную систему:
$$3{,}5_{10} = 11{,}1_2.$$

Поскольку мантисса должна быть нормализованной, то есть иметь целую часть, равную $1$, перенесём запятую на одну позицию влево и домножим число на $2^1$:
$$11{,}1_2
= 1{,}{\color{Cerulean} 11}_2 \cdot 2^{\color{YellowGreen} 1}
= 1{,}75 \cdot 2^{1}.$$

В побитовое представление переносится только {\color{Cerulean}дробная часть мантиссы}, дополненная справа нулями.

Порядок --- показатель степени у числа 2, он записывается смещённым на 127:
$$e^*
= {\color{YellowGreen} e} + 127
= {\color{YellowGreen} 1} + 127 = 128_{10}
= {\color{ForestGreen} 10000000}_2.$$
$$N_1
= (-1)^{\color{Maroon} 0} \cdot 1{,}{\color{Cerulean}11}
\cdot 2^{{\color{YellowGreen} 1}}
= 1{,}75 \cdot 2^{1}.$$

\answer{1}{\fp{0}{10000000}{11000000000000000000000}}

%%%%%%%%%%%%%%%%%%%%%%%%%%%%%%%%%%%%%%%%%%%%%%%%%%%%%%%%%%%%%%%%%%%%%%%%%%%%%%%%

Дано побитовое представление числа $N_2$:

\fp{0}{10000011}{11010010000000000000000}

Запишем его в общей форме и в десятичной системе. Поскольку первый бит равен {\color{Maroon} 0}, число положительное:
$(-1)^{\color{Maroon} 0} = +1$.

Целая часть нормализованной мантиссы равна единице; хотя и не записывается в побитовом представлении, она всегда присутствует. Нормализованная мантисса равна:
$$m = 1{,}{\color{Cerulean} 1101001}_2
= 1{,}8203125_{10}.$$

Экспонента хранится смещённой на 127, следовательно:
$$
{\color{YellowGreen} e}
= {\color{ForestGreen} e^*} - 127
= {\color{ForestGreen} 10000011}_2 - 127_{10}
= 131_{10} - 127_{10}
= {\color{YellowGreen} 4}.$$
$$N_2 = (-1)^{\color{Maroon} \answer{2a}{0}}
\cdot 1{,}{\color{Cerulean}\answer{2b}{8203125}}
\cdot 2^{\color{YellowGreen} \answer{2c}{4}}
= \answer{2}{29{,}125}.$$

%%%%%%%%%%%%%%%%%%%%%%%%%%%%%%%%%%%%%%%%%%%%%%%%%%%%%%%%%%%%%%%%%%%%%%%%%%%%%%%%

Найдём произведение этих двух чисел:
$$N_1 \cdot N_2
= 1{,}75 \cdot 2^1 \cdot 1{,}8203125 \cdot 2^4 =$$
$$= 3{,}185546875 \cdot 2^{1 + 4} =$$
$$= 3{,}185546875 \cdot 2^5 =$$
$$= \answer{3a}{101{,}9375}.$$

Проверка с помощью калькулятора показывает, что вычисления верные.

Бит знака $s = {\color{Maroon} 0}$, поскольку число положительное.

Мантисса: $
3{,}185546875_{10}
= 11{,}001011111_2
$.
Нормализованная мантисса:
$$m = 1{,}{\color{Cerulean}1001011111}_2 \cdot 2^{{\color{YellowGreen} 1}}.$$

Поскольку нормализованная мантисса содержит $2^{{\color{YellowGreen} 1}}$, порядок увеличивается на 1, теперь он равен $e = {\color{YellowGreen} 6}$. Смещённый порядок:
$$e^*
= e + 127
= {\color{YellowGreen} 6} + 127
= 133_{10}
= {\color{ForestGreen} 10000101}_2.$$

Представление числа с плавающей запятой:
$$N_1 \cdot N_2
= 101{,}9375
= (-1)^{{\color{Maroon} 0}}
\cdot 1{,}{\color{Cerulean} 1001011111}
\cdot 2^{{\color{YellowGreen} 6}}$$

Побитовое представление:

\answer{3b}{\fp{0}{10000101}{10010111110000000000000}}

\newpage

\section{\MakeUppercase{Заключение}}

В ходе работы мы ознакомились с представлением в памяти компьютера чисел с плавающей запятой. Мы выполнили поставленные задачи: закодировали числа в формат с плавающей запятой, декодировали второе число, умножили их и проверили результат.

\end{document}
